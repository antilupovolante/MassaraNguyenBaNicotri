\chapter{Introduction}

\section{Purpose}
In the last years, more and more students are becoming interested in programming and many educators have realised 
the need of new innovative methods to improve coding skills. \\
CodeKataBattle aims to assist students in enhancing their programming skills by challenging them with creative tasks
 in a competitive and stimulating environment. \\
 \\
 The following document want to describe the system focusing on the requirements and specification, providing scenarios 
 and use case to specify what the system must do and how it should interact with the stakeholders.\\  

 \subsection{Goals}
 In the following table we describe the main goals that our system want to achive.\\
 \begin{table}[H]
    \resizebox{0.8\textwidth}{!}{%
    \begin{tabular}{|c|l|}
    \hline
    \rowcolor[HTML]{b8c8d5}
    \multicolumn{1}{|c|}{\cellcolor[HTML]{b8c8d5}\textbf{Goal}} & \multicolumn{1}{c|}{\cellcolor[HTML]{b8c8d5}\textbf{Description}} \\ \hline
    G.1 \label{G.1}& \begin{tabular}[c]{@{}l@{}} Allow students to compete in a tournament \end{tabular}  \\ \hline
    G.2 \label{G.2}& Allows educators to create challenges for students. \\ \hline
    G.3 \label{G.3}& Allows educators to grade students' projects. \\ \hline
    G.4 \label{G.4}& Allows students to collect badges. \\ \hline
    \end{tabular}
    }
    \end{table}
    \clearpage
\section{Scope}
The main actors of the system are students and educator. Educator can:
\begin{itemize}
    \item \textbf{Create a tournament:} decide which colleague can create battles within the tournament and defines badges 
    that represent the achievements of individual students; 
    \item \textbf{Create a battle:} set configurations and rules for that battle;
    \item \textbf{Evaluate:} manually assign a personal score to the students' works.
\end{itemize}

Students can:
\begin{itemize}
    \item \textbf{Join a tournament};
    \item \textbf{Partecipate on a battle:} create a team and complete the project with his code;
    \item \textbf{Collect badges:} based on the rules of the tournament and his performance.
\end{itemize}

\subsection{World Phenomena}

\begin{table}[H]
    \resizebox{0.8\textwidth}{!}{%
    \begin{tabular}{|c|l|}
    \hline
    \rowcolor[HTML]{b8c8d5}
    \multicolumn{1}{|c|}{\cellcolor[HTML]{b8c8d5}\textbf{World Phenomenon}} & \multicolumn{1}{c|}{\cellcolor[HTML]{b8c8d5}\textbf{Description}} \\ \hline
    WP.1 \label{WP.1}& Educator wants to create a tournament.  \\ \hline
    WP.2 \label{WP.2}& Educator wants to create a new battle. \\ \hline
    WP.3 \label{WP.3}& A student want to join a tournament.   \\ \hline
    WP.4 \label{WP.4}& A student wants to join a battle.  \\ \hline
    WP.5 \label{WP.5}& An educator evaluates the works done by students.   \\ \hline
    \end{tabular}
    }
    \end{table}
    \clearpage

\subsection{Shared Phenomena}

\begin{table}[H]
    \resizebox{0.8\textwidth}{!}{%
    \begin{tabular}{|c|l|c|}
    \hline
    \rowcolor[HTML]{b8c8d5}
    \multicolumn{1}{|c|}{\cellcolor[HTML]{b8c8d5}\textbf{Shared Phenomenon}} & \multicolumn{1}{c|}{\cellcolor[HTML]{b8c8d5}\textbf{Description}}& \multicolumn{1}{c|}{\cellcolor[HTML]{b8c8d5}\textbf{Controlled by}} \\ \hline
    SP.1 \label{SP.1}&  --- & --- \\ \hline
    SP.2 \label{SP.2}&   --- & ---  \\ \hline
    SP.3 \label{SP.3}&   --- & ---   \\ \hline
    SP.4 \label{SP.4}&   --- & ---   \\ \hline
    SP.5 \label{SP.5}&   --- & ---   \\ \hline
    \end{tabular}
    }
    \end{table}
    \clearpage


\section{Definitions, Acronyms, Abbreviations}

\section{Revision History}

\section{Reference Documents}

\section{Document Structure}
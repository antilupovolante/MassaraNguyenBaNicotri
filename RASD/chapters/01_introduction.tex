\chapter{Introduction}

\section{Purpose}
In the last years, more and more students are becoming interested in programming and many educators have realised 
the need of new innovative methods to improve coding skills. \\
CodeKataBattle aims to assist students in enhancing their programming skills by challenging them with creative tasks
 in a competitive and stimulating environment. \\
 \\
 The following document want to describe the system focusing on the requirements and specification, providing scenarios 
 and use case to specify what the system must do and how it should interact with the stakeholders.\\  

 \subsection{Goals}
 In the following table we describe the main goals that our system want to achive.\\
 \begin{table}[H]
    \centering
    \resizebox{0.8\textwidth}{!}{%
    \begin{tabular}{|c|l|}
    \hline
    \rowcolor[HTML]{b8c8d5}
    \multicolumn{1}{|c|}{\cellcolor[HTML]{b8c8d5}\textbf{Goal}} & \multicolumn{1}{c|}{\cellcolor[HTML]{b8c8d5}\textbf{Description}} \\ \hline
    G.1 \label{G.1}& \begin{tabular}[c]{@{}l@{}} Allow students to compete in a tournament \end{tabular}  \\ \hline
    G.2 \label{G.2}& Allows educators to create challenges for students. \\ \hline
    G.3 \label{G.3}& Allows educators to grade students' projects. \\ \hline
    G.4 \label{G.4}& Allows students to collect badges. \\ \hline
    \end{tabular}
    }
    \end{table}
    \clearpage
\section{Scope}
The main actors of the system are students and educator. Educator can:
\begin{itemize}
    \item \textbf{Create a tournament:} decide which colleague can create battles within the tournament and defines badges 
    that represent the achievements of individual students; 
    \item \textbf{Create a battle:} set configurations and rules for that battle;
    \item \textbf{Evaluate:} manually assign a personal score to the students' works.
\end{itemize}

Students can:
\begin{itemize}
    \item \textbf{Join a tournament};
    \item \textbf{Partecipate on a battle:} create a team and complete the project with his code;
    \item \textbf{Collect badges:} based on the rules of the tournament and his performance.
\end{itemize}

\subsection{World Phenomena}

\begin{table}[H]
    \centering
    \resizebox{0.8\textwidth}{!}{%
    \begin{tabular}{|c|l|}
    \hline
    \rowcolor[HTML]{b8c8d5}
    \multicolumn{1}{|c|}{\cellcolor[HTML]{b8c8d5}\textbf{W.P.}} & \multicolumn{1}{c|}{\cellcolor[HTML]{b8c8d5}\textbf{Description}} \\ \hline
    WP.1 \label{WP.1}& Educator wants to create a tournament.  \\ \hline
    WP.2 \label{WP.2}& Educator wants to create a new battle. \\ \hline
    WP.3 \label{WP.3}& A student want to join a tournament.   \\ \hline
    WP.4 \label{WP.4}& A student wants to join a battle.  \\ \hline
    WP.5 \label{WP.5}& An educator evaluates the works done by students.   \\ \hline
    \end{tabular}
    }
    \end{table}
    \clearpage

\subsection{Shared Phenomena}

\begin{table}[H]
    \centering
    \resizebox{\textwidth}{!}{%
    \begin{tabular}{|c|l|c|}
    \hline
    \rowcolor[HTML]{b8c8d5}
    \multicolumn{1}{|c|}{\cellcolor[HTML]{b8c8d5}\textbf{S.P.}} & \multicolumn{1}{c|}{\cellcolor[HTML]{b8c8d5}\textbf{Description}}& \multicolumn{1}{c|}{\cellcolor[HTML]{b8c8d5}\textbf{Controlled by}} \\ \hline
    SP.1 \label{SP.1}&  The system notifies the student about upcoming battles. & Machine \\ \hline
    SP.2 \label{SP.2}&  The student commits his code. & World  \\ \hline
    SP.3 \label{SP.3}&  The educator configures the tournament rules. & World   \\ \hline
    SP.4 \label{SP.4}&  The student forms a team. & World   \\ \hline
    SP.5 \label{SP.5}&  \begin{tabular}[c]{@{}l@{}} The educator grants other colleagues the permission to \\create battles within a tournament. \end{tabular} & World  \\ \hline
    SP.6 \label{SP.6}&  The educator configures the battle. & World  \\ \hline
    SP.7 \label{SP.7}&  The student joins a team. & World  \\ \hline
    SP.8 \label{SP.8}&  The system creates a GitHub repository. & Machine  \\ \hline
    SP.9 \label{SP.9}&  The student forks and sets up the GitHub repository. & World  \\ \hline
    SP.10 \label{SP.10}&  GitHub Actions notifies the system about students' commits. & World  \\ \hline
    SP.11 \label{SP.11}&  The system shows the battle score of the team. & Machine  \\ \hline
    SP.12 \label{SP.12}&  The system notifies when the final battle rank becomes available. & Machine  \\ \hline
    SP.13 \label{SP.13}&  The system shows the tournament rank. & Machine  \\ \hline
    SP.14 \label{SP.14}&  The system shows the list of ongoing tournaments. & Machine  \\ \hline
    SP.15 \label{SP.15}&  The system shows the student's badges. & Machine  \\ \hline
    SP.16 \label{SP.16}&  The educator defines the badges of a tournament. & World  \\ \hline
    SP.17 \label{SP.17}&  All users can visualize the profile of a user. & World  \\ \hline
    \end{tabular}
    }
    \end{table}
    \clearpage


\section{Definitions, Acronyms, Abbreviations}
\subsection{Definitions}
\begin{table}[H]
    \centering
    \begin{tabular}{|c|l|}
    \hline
    \rowcolor[HTML]{B8C8D5} 
    \textbf{Definitions} & \multicolumn{1}{c|}{\cellcolor[HTML]{B8C8D5}\textbf{Meaning}} \\ \hline
    Code Kata & \begin{tabular}[c]{@{}l@{}}A code kata battle is a programming exercise in a programming language \\
        of choice. The exercise includes a brief textual description and a software \\
        project with build automation scripts that contains a set of test cases that the \\
        program must pass, but without the program implementation. \end{tabular}  \\ \hline
    \end{tabular}
\end{table}


\subsection{Acronyms}

\begin{table}[H]
    \centering
    \begin{tabular}{|c|l|}
    \hline
    \rowcolor[HTML]{B8C8D5} 
    \textbf{Acronyms} & \multicolumn{1}{c|}{\cellcolor[HTML]{B8C8D5}\textbf{Meaning}} \\ \hline
    CKB & CodeKataBattle  \\ \hline
    \end{tabular}
\end{table}

\subsection{Abbreviations}

\begin{table}[H]
    \centering
    \begin{tabular}{|c|l|}
    \hline
    \rowcolor[HTML]{B8C8D5} 
    \textbf{Abbreviations} & \multicolumn{1}{c|}{\cellcolor[HTML]{B8C8D5}\textbf{Meaning}} \\ \hline
    WP & World Phenomena  \\ \hline
    SP & Shared Phenomena \\ \hline
    G  & Goal             \\ \hline
    R  & Requirement             \\ \hline
    NFR  & Non Functional Requirement             \\ \hline
    D  & Domain Assumption             \\ \hline
    w.r.t. & with reference to \\ \hline
    e.g. & exempli gratia \\ \hline
    i.e. & id est \\ \hline
    etc. & etcetera \\ \hline
    \end{tabular}
\end{table}

\section{Revision History}

\section{Reference Documents}
\begin{itemize}
    \item Course slides on WeBeep. 
    \item RASD assignment document.
\end{itemize}


\section{Document Structure}
The structure of this RASD document is the following:
\begin{enumerate}
    \item \textbf{Introduction}: In this section is presented the purpose of this document highlighting in particular the main goals, the audience which is referred to, the identification of the product and application domain  describing world and shared phenomena and, lastly, the terms definitions.
    \item \textbf{Overall Description}: This chapter describes the possible scenarios of the platform, the shared phenomena presented at the beginning of the document and assumptions on the domain of the application.
    \item \textbf{Specific Requirements}: Includes all the requirements in a more specific way than the "Overall Description" section. Moreover, it is useful to show functional requirements in terms of use cases diagrams, sequence/activity diagrams.
    \item \textbf{Formal Analysis Using Alloy}: Includes Alloy models which are used for the description of the application domain and his properties, referring to the operations which the system has to provide.
    \item \textbf{Effort Spent}: This section shows the effort spent in terms of time for each team member and the whole team.
    \item \textbf{References}: Includes all documents that were helpful in drafting the RASD.
\end{enumerate}
\chapter{Architectural Design}
This section aims to present and analyze the architecture of the S2B in a top-down manner. 
We discuss about the architectural design choices and the reasons behind them. 

\section{Overview: High-level components and their interactions}
The figure shown below represents a high-level description of the components which make up the system.
\begin{figure}[H]
    \centering
    \includegraphics[width=\textwidth]{images/component_view/high_level.png}
    \caption{Overview CKB architecture}
    \label{fig:CKB Architecture}
\end{figure}
A web interface will be used to access the platform. 
The overall architecture of the system is based on a three-tier architecture, 
with the application servers interacting with a database management system and
using APIs to retrieve and store data. \\
The three logical layers corrspond to three different physical layers and each layer can communicate only with the adjacent ones.
The interactions between clients and server are stateless according to the REST architectural style. \\
The web server is responsible for the communication with the clients and for the management of the requests. \\
The application server holds the business logic of the application and it can communicate with the database server to retrieve and store data, also with the GitHub platform and the Email Service. 

\section{Component view}
\begin{figure}[H]
    \centering
    \includegraphics[width=\textwidth]{images/component_view/Component_view.png}
    \caption{High level component view}
\end{figure}

\begin{figure}[H]
    \centering
    \includegraphics[width=\textwidth]{images/component_view/CKB_component.png}
    \caption{CKB server component view}
\end{figure}

\begin{figure}[H]
    \centering
    \includegraphics[width=\textwidth]{images/component_view/Auth.png}
    \caption{Auth component view}
\end{figure}

\section{Deployment view}
The figure below shows the architecture of the system. All the users access to the WebApp through the browser, which communicate with
the Web Server. Both Web Server and Application Server are hosted on a Cloud Provider. This choice offers many
advantages, such as:

\begin{itemize}
    \item  \textbf{Scalability and flexibility: }the ability of adding and removing resources efficiently throw 
    the use of load balancing services which allows the application server to manage traffic and workload. 
    \item   \textbf{Security: }the ability to protect the application server using firewall and DMZ, againist cyberattacks and possible threats.
    \item   \textbf{Cost efficiency:} the ability to pay only for the resources used which can help to lower the overall cost. 
\end{itemize}

\begin{figure}[H]
    \centering
    \includegraphics[width=1\textwidth]{images/Deployment_diagram.png}
    \caption{Deployment diagram}
\end{figure}

The deployment diagram offers a more detailed view over the hardware and software components of the system. 
\begin{itemize}
    \item  \textbf{PC: }is any device having a browser capable of running the JavaScript code.
    \item   the Cloud Services will host all the business and data logic for the system. It is characterized by 
    \begin{itemize}
        \item  \textbf{Firewall:} devices are used to protect and filter incoming connections to the logic and data layers of the system. It protects the system from unauthorized access and malicious attacks.
        \item  \textbf{Load Balancer:} services are used to distribute the workload across multiple servers. It helps to improve the performance and reliability of the system. It also helps to ensure that application can handle a large volume of requests, without any downtime.
        \item  \textbf{Multiple copies of the application:} are used to ensure that the system is always available. The different instances can be created and destroyed dynamically, based on the workload. It also helps to achieve fault tolerance by allowing traffic to be redirect to a different instance, if one instance becomes anavailable.
    \end{itemize}
    \item   \textbf{Database:} is used to store all the data of the system. It uses MySQL as DBMS to retrieve and store data.
    \item   \textbf{Mail Provider: }is used to send notifications to the users. It uses Gmail as mail provider.
\end{itemize}

\section{Runtime view}
\section{Component interfaces}
\section{Selected architetural styles and patterns}
\section{Other design decisions}
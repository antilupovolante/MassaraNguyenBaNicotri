\chapter{Implementation, integration and testing plan}
This chapter presents the order of implementation of subsystems and components, 
the integration strategy and the testing plan.\\
Subsystems are developed and integrated using a bottom-up approach, starting from the 
individual parts and then linking them together.\\

\section{Implementation Plan}
The implementation of the system is realized following a bottom-up method, with a particular 
emphasis on the back-end development since users will interact with the system only through 
its APIs.\\
Particularly, the implementation order is the following:
\begin{enumerate}
    \item \textbf{DBMS, DataManager:} this is the first part to be implemented since it is responsible 
    for storing and accessing the data needed by the system.
    \item \textbf{AuthManager:} this component is responsible for users authentication and authorization, 
    which are key aspects of the system, and therefore is heavily linked to the DBMS and DataManager.
    \item \textbf{MailManager, EvaluationTool:} these components can be implemented in parallel since they are 
    independent from each other. In this phase we need to choose the programming languages that the system will
    support and consequently the static analysis tools that will be integrated into the system.    
    \item \textbf{CKBHandler:} this component is responsible for linking together the previous components, so 
    it must be developed after them.
    \item \textbf{UserApp:} this component is responsible for the interaction with the user, so it is to be 
    developed after the back-end has been completed.
\end{enumerate}
\section{Integration Plan}

\begin{figure}[H]
    \centering
    \includegraphics[width=0.2\textwidth]{images/integration_plan/first_level_integration.png}
    \caption{First level of integration}
\end{figure}

\begin{figure}[H]
    \centering
    \includegraphics[width=0.45\textwidth]{images/integration_plan/second_level_integration.png}
    \caption{Second level of integration}
\end{figure}

\begin{figure}[H]
    \centering
    \includegraphics[width=1\textwidth]{images/integration_plan/third_level_integration.png}
    \caption{Third level of integration}
\end{figure}

\begin{figure}[H]
    \centering
    \includegraphics[width=1\textwidth]{images/integration_plan/fourth_level_integration.png}
    \caption{Fourth level of integration}
\end{figure}

\begin{figure}[H]
    \centering
    \includegraphics[width=1\textwidth]{images/integration_plan/fifth_level_integration.png}
    \caption{Fifth level of integration}
\end{figure}


\section{System Testing}
Testing should be performed both during and after the development of the system to ensure that all requirements 
are satisfied.\\
The testing plan is divided in two parts: unit testing and integration testing. Integration testing should also 
focus on the security aspects of the system to avoid any possible security breach.\\
Alpha testing should be performed whenever a new feature is implemented to receive feedback from the stakeholders 
regarding their level of satisfaction and eventual malfunctions.\\
A beta test phase could take place after the development of the system to ensure that the system is ready for 
a real-world usage and to collect information about performance and eventual scalability issues.\\
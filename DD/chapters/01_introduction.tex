\chapter{Introduction}

\section{Purpose}
In the last years, more and more students are becoming interested in programming and many educators have realised 
the need of new innovative methods to improve coding skills. \\
CodeKataBattle aims to assist students in enhancing their programming skills by challenging them with creative tasks
 in a competitive and stimulating environment. \\
 \\
 The following document wants to describe the system focusing on the design choiches, descripting at a high level the technologies
  and components used to implement the platform, paying attention also to the interactive behaviour of system users.\\
  
\section{Scope}
CodeKataBattle is a web application that allows students to compete in tournaments created by educators. During a tournament, educators 
can create battles and students can join them. During a battle, students can create a team and complete the project with their code, which 
will be evaluated by the system and optionally by the educators. Battle scores contribute to both battle and tournament ranks and at the end 
of a tournament students can also collect badges based on their performance.\\
For a more detailed description of the features that the system offers to end users, please refer to the RASD.\\

\section{Definitions, Acronyms, Abbreviations}
\subsection{Definitions}
\begin{table}[H]
    \centering
    \begin{tabular}{|c|l|}
    \hline
    \rowcolor[HTML]{B8C8D5} 
    \textbf{Definitions} & \multicolumn{1}{c|}{\cellcolor[HTML]{B8C8D5}\textbf{Meaning}} \\ \hline
    Code Kata & \begin{tabular}[c]{@{}l@{}}A code kata battle is a programming exercise in a programming\\language 
        of choice. The exercise includes a brief textual \\description and a software 
        project with build automation scripts \\that contains a set of test cases that the 
        program must pass, \\but without the program implementation. \end{tabular}  \\ \hline
    Tournament & \begin{tabular}[c]{@{}l@{}}A tournament is a set of battles. It is created by an educator \\and it is 
        composed by a set of rules and possible badges. \end{tabular}  \\ \hline
    Battle & \begin{tabular}[c]{@{}l@{}}A battle is a competition between teams. It is created by an\\ educator 
        and it is composed by a set of rules. \end{tabular}  \\ \hline
    Team & \begin{tabular}[c]{@{}l@{}}A team is a group of students that compete in a battle. \end{tabular}  \\ \hline
    Badge & \begin{tabular}[c]{@{}l@{}}A badge is a reward that a student can obtain by participating\\in a 
        tournament. \end{tabular}  \\ \hline
    Application User Interface & \begin{tabular}[c]{@{}l@{}}The application user interface defines the operations that can\\ be 
        invoked on  the component which offers it. \end{tabular}  \\ \hline
    Demilitarized zone & \begin{tabular}[c]{@{}l@{}}A demilitarized zone is a middle ground between an \\organization's trusted 
        internal network and an untrusted, \\external one. \end{tabular}  \\ \hline
    \end{tabular}
\end{table}

\subsection{Acronyms}
\begin{table}[H]
    \centering
    \begin{tabular}{|c|l|}
    \hline
    \rowcolor[HTML]{B8C8D5} 
    \textbf{Acronyms} & \multicolumn{1}{c|}{\cellcolor[HTML]{B8C8D5}\textbf{Meaning}} \\ \hline
    CKB & CodeKataBattle  \\ \hline
    UI & User Interface  \\ \hline
    RASD & Requirements Analysis and Specification Document  \\ \hline
    DD & Design Document  \\ \hline
    S2B & System To Be \\ \hline
    \end{tabular}
\end{table}

\subsection{Abbreviations}
\begin{table}[H]
    \centering
    \begin{tabular}{|c|l|}
    \hline
    \rowcolor[HTML]{B8C8D5} 
    \textbf{Abbreviations} & \multicolumn{1}{c|}{\cellcolor[HTML]{B8C8D5}\textbf{Meaning}} \\ \hline
    WP & World Phenomena  \\ \hline
    SP & Shared Phenomena \\ \hline
    G  & Goal             \\ \hline
    R  & Requirement             \\ \hline
    UC  & Use Case             \\ \hline
    w.r.t. & with reference to \\ \hline
    e.g. & exempli gratia \\ \hline
    i.e. & id est \\ \hline
    etc. & etcetera \\ \hline
    DB & Database \\ \hline
    API & Application Programming Interface \\ \hline
    DMZ & Demilitarized Zone \\ \hline
    \end{tabular}
\end{table}

\section{Revision history}

\section{Reference Documents}
\begin{itemize}
    \item Course slides on WeBeep. 
    \item RASD and DD assignment document.
    \item CodeKataBattle Requirements Analysis and Specification Document.
\end{itemize}

\section{Document Structure}
The document is structured as follows:
\begin{itemize}
    \item \textbf{Introduction}: The first chapter includes the introduction in which is explained the purpose of the document, then, s brief recall of the concepts introduced in the RASD is given.
    Finally, important information for the readers is given, i.e. definitions, acronyms, synonyms and the set of reference documents.
    \item \textbf{Architectural Design}: This chapter includes the description of the system architecture, the design choices and the interaction between the components. It includes a component view, a deployment view and a runtime view.
    \item \textbf{User Interface Design}: This chapter includes the description of the user interface design, with mockups and UX diagrams.
    \item \textbf{Requirements Traceability}: This chapter includes the mapping between the requirements defined in the RASD and the design elements specified in the DD.
    \item \textbf{Implementation, Integration and Test Plan}: This chapter includes the description of the implementation, integration and test plan to which developers have to stick in order to produce the correct system in a correct way.
    \item \textbf{Effort Spent}: This chapter includes the description of the effort spent by each member of the group.
    \item \textbf{References}: This chapter includes the list of the documents used to write this document.
\end{itemize}